\documentclass[../main.tex]{subfiles}

\begin{document}

\subsection{Social impacts}
The social impacts of this project are primarily related to science. Research groups in the field of structural biology may be positively affected by the advances proposed on this work. Firstly, the results presented here show that the algorithm leads to superior results when it is used in conjunction with current solutions. In addition, this same tandem performs better in terms of computational time. As a consequence, the project has the potential to increase the productivity of structural biologists by providing then with better results faster. This increased productivity of researchers leads to faster vaccine and drug developments, which has implications in the pharmaceutical and healthcare industry.

In addition, the project has been developed on an academic environment under a \gls{foss} licensing terms. This means that it can be used as a basis for further improvements or new fields of applications.   

\subsection{Environmental impacts}
As mentioned earlier, one of the advances of the algorithm relate to the lower computational cost associated to the 3D classification process. This directly relates into a reduction of the energy consumption of the compute infrastructure.

\subsection{Economic impacts}
As a consequence of the previous arguments, the project poses a notable economic impact on research facilities. Firstly, the fact that this algorithm is distributed under a \gls{foss} lices means that its utilization comes at no cost. Secondly, the project reduction in power consumption directly translates into lower power bills. Consequently, these two factors combine to decrease the operational expenses for various research groups.

\end{document}